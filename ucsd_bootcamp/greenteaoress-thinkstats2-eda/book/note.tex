\documentclass[12pt]{article}

\usepackage{mathtools}

\title{Bayes's theorem and logistic regression}
\author{Allen B. Downey}

\newcommand{\logit}{\mathrm{logit}}
\renewcommand{\P}{\mathrm{P}}
\renewcommand{\O}{\mathrm{O}}
\newcommand{\LR}{\mathrm{LR}}
\newcommand{\LO}{\mathrm{LO}}
\newcommand{\LLR}{\mathrm{LLR}}
\newcommand{\OR}{\mathrm{OR}}
\newcommand{\LOR}{\mathrm{LOR}}
\newcommand{\IF}{\mathrm{if}}
\newcommand{\notH}{\neg H}

\setlength{\headsep}{3ex}
\setlength{\parindent}{0.0in}
\setlength{\parskip}{1.7ex plus 0.5ex minus 0.5ex}

\begin{document}

\maketitle

\begin{abstract}
My two favorite topics in probability and statistics are
Bayes's theorem and logistic regression.  Because there are
similarities between them, I have always assumed that there is
a connection.  In this note, I demonstrate the
connection mathematically, and (I hope) shed light on the
motivation for logistic regression and the interpretation of
the results.
\end{abstract}


\section{Bayes's theorem}

I'll start by reviewing Bayes's theorem, using an example that came up
when I was in grad school.  I signed up for a class on Theory of
Computation.  On the first day of class, I was the first to arrive.  A
few minutes later, another student arrived.  Because I was expecting
most students in an advanced computer science class to be male, I was
mildly surprised that the other student was female.  Another female
student arrived a few minutes later, which was sufficiently
surprising that I started to think I was in the wrong room.  When
another female student arrived, I was confident I was in the wrong
place (and it turned out I was).

As each student arrived, I used the observed data to update my
belief that I was in the right place.  We can use Bayes's theorem to
quantify the calculation I was doing intuitively.

I'll us $H$ to represent the hypothesis that I was in the right
room, and $F$ to represent the observation that the first other
student was female.  Bayes's theorem provides an algorithm for
updating the probability of $H$:

\[ \P(H|F) = \P(H)~\frac{\P(F|H)}{P(F)}\]

Where

\begin{itemize}

\item $\P(H)$ is the prior probability of $H$ before the other
student arrived.

\item $\P(H|F)$ is the posterior probability of $H$, updated based
on the observation $F$.

\item $\P(F|H)$ is the likelihood of the data, $F$, assuming that
the hypothesis is true.

\item $P(F)$ is the likelihood of the data, independent of $H$.
 
\end{itemize}

Before I saw the other students, I was confident I was in the right
room, so I might assign $\P(H)$ something like 90\%.

When I was in grad school most advanced computer science classes were
90\% male, so if I was in the right room, the likelihood of the
first female student was only 10\%.  And the likelihood of three
female students was only 0.1\%.

If we don't assume I was in the right room, then the likelihood of
the first female student was more like 50\%, so the likelihood
of all three was 12.5\%.

Plugging those numbers into Bayes's theorem yields $\P(H|F) = 0.64$
after one female student, $\P(H|FF) = 0.26$ after the second,
and $\P(H|FFF) = 0.07$ after the third.


\section{Logistic regression}

Logistic regression is based on the following functional form:

\[ \logit(p) = \beta_0 + \beta_1 x_1 + ... + \beta_n x_n \]

where the dependent variable, $p$, is a probability,
the $x$s are explanatory variables, and the $\beta$s are
coefficients we want to estimate.  The $\logit$ function is the
log-odds, or

\[ \logit(p) = \ln \left( \frac{p}{1-p} \right) \]

When you present logistic regression like this, it raises
three questions:

\begin{itemize}

\item Why is $\logit(p)$ the right choice for the dependent
variable?

\item Why should we expect the relationship between $\logit(p)$
and the explanatory variables to be linear?

\item How should we interpret the estimated parameters?

\end{itemize}

The answer to all of these questions turns out to be Bayes's
theorem.  To demonstrate that, I'll use a simple example where
there is only one explanatory variable.  But the derivation
generalizes to multiple regression.

On notation: I'll use $\P(H)$ for the probability
that some hypothesis, $H$, is true.  $\O(H)$ is the odds of the same
hypothesis, defined as

\[ \O(H) = \frac{\P(H)}{1 - \P(H)} \]

I'll use $\LO(H)$ to represent the log-odds of $H$:

\[ \LO(H) = \ln \O(H) \]

I'll also use $\LR$ for a likelihood ratio, and $\OR$ for an odds
ratio.  Finally, I'll use $\LLR$ for a log-likelihood ratio, and
$\LOR$ for a log-odds ratio.





\section{Making the connection}

To demonstrate the connection between Bayes's theorem and
logistic regression, I'll start with the odds form
of Bayes's theorem.  Continuing the previous example,
I could write

\begin{equation} \label{A}
\O(H|F) = \O(H)~\LR(F|H)
\end{equation}

where

\begin{itemize}

\item $\O(H)$ is the prior odds that I was in the right room,

\item $\O(H|F)$ is the posterior odds after seeing one female student,

\item $\LR(F|H)$ is the likelihood ratio of the data, given
the hypothesis.

\end{itemize}

The likelihood ratio of the data is:

\[ \LR(F|H) = \frac{\P(F|H)}{\P(F|\notH)} \]

where $\notH$ means $H$ is false.

Noticing that logistic regression is expressed in terms of
log-odds, my next move is to write the log-odds form of
Bayes's theorem by taking the log of Eqn~\ref{A}:

\begin{equation} \label{B}
\LO(H|F) = \LO(H) + \LLR(F|H)
\end{equation}

If the first student to arrive had been male, we would write

\begin{equation} \label{C} \nonumber
\LO(H|M) = \LO(H) + \LLR(M|H)
\end{equation}

Or more generally if we use $X$ as a variable to represent
the sex of the observed student, we would write

\begin{equation} \label{D}
\LO(H|X) = \LO(H) + \LLR(X|H)
\end{equation}

I'll assign $X=0$ if the observed student is female and
$X=1$ if male.  Then I can write:

\begin{equation} \label{E} \nonumber
\LLR(X|H) = \left\{
  \begin{array}{lr}
    \LLR(F|H) & \IF ~X = 0\\
    \LLR(M|H) & \IF ~X = 1
  \end{array}
\right.
\end{equation}

Or we can collapse these two expressions into one by using
$X$ as a multiplier:

\begin{equation} \label{F}
\LLR(X|H) = \LLR(F|H) + X [\LLR(M|H) - \LLR(F|H)]
\end{equation}


\section{Odds ratios}

The next move is to recognize that 
the part of Eqn~\ref{F} in brackets is the log-odds ratio
of $H$.  To see that, we need to look more closely at odds ratios.

Odds ratios are often used in medicine to describe the association
between a disease and a risk factor.  In the example scenario, we
can use an odds ratio to express the odds of the hypothesis
$H$ if we observe a male student, relative to the odds if we
observe a female student:

\[ \OR_X(H) = \frac{\O(H|M)}{\O(H|F)} \]

I'm using the notation $\OR_X$ to represent the odds ratio
associated with the variable $X$.

Applying Bayes's theorem to
the top and bottom of the previous expression yields

\[ \OR_X(H) = \frac{\O(H)~\LR(M|H)}{\O(H)~\LR(F|H)} = 
\frac{\LR(M|H)}{\LR(F|H)}\]

Taking the log of both sides yields

\begin{equation} \label{G}
\LOR_X(H) = \LLR(M|H) - \LLR(F|H)
\end{equation}

This result should look familiar, since it appears in
Eqn~\ref{F}.


\section{Conclusion}

Now we have all the pieces we need; we just have to assemble them.
Combining Eqns~\ref{F} and \ref{G} yields  

\begin{equation} \label{H}
\LLR(H|X) = \LLR(F) + X~\LOR(X|H)
\end{equation}

Combining Eqns~\ref{D} and \ref{H} yields

\begin{equation} \label{I}
\LO(H|X) = \LO(H) + \LLR(F|H) + X~\LOR(X|H)
\end{equation}

Finally, combining Eqns~\ref{B} and \ref{I} yields

\[ \LO(H|X) = \LO(H|F) + X~\LOR(X|H) \]

We can think of this equation as the log-odds form of Bayes's theorem,
with the update term expressed as a log-odds ratio.  Let's compare
that to the functional form of logistic regression:

\[ \logit(p) = \beta_0 + X \beta_1 \]

The correspondence between these equations suggests the following
interpretation:

\begin{itemize}

\item The predicted value, $\logit(p)$, is the posterior log
odds of the hypothesis, given the observed data.

\item The intercept, $\beta_0$, is the log-odds of the
hypothesis if $X=0$.

\item The coefficient of $X$, $\beta_1$, is a log-odds ratio
that represents odds of $H$ when $X=1$, relative to
when $X=0$.

\end{itemize}

This relationship between logistic regression and Bayes's theorem
tells us how to interpret the estimated coefficients.  It also
answers the question I posed at the beginning of this note:
the functional form of logistic regression makes sense because
it corresponds to the way Bayes's theorem uses data to update
probabilities.

\end{document}
